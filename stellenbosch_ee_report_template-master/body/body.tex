\graphicspath{{body/fig/}}

\chapter{Body}
\label{chap:body}

\section{Model Definition}

\begin{figure}[!h]
    \centering
    \includegraphics[width=\linewidth]{model-drawing.png}
    \caption[I am the short caption that appears in the list of figures, without references.]{ I am a caption below the figure of course}
    \label{fig:model-drawing}
\end{figure}

This is a reference to Figure~\ref{fig:model-drawing}

RVs:
\begin{itemize}
    \item Hidden Drought State RVs $\equiv S_t = \{1,2,\dots,m\}$    
    \item Attribute RVs $\equiv A^{(n)}_t = \{1,2,\dots,C_n\}$
\end{itemize}

Some further notation:
\begin{itemize}
    \item $\vec{S}_{1:T}  = \{S_1,S_2,\dots,S_T\}$    
    \item $A_{1:T} = \{\vec{A}_1,\vec{A}_2,\dots,\vec{A}_T\}$
    \begin{itemize}
        \item Where $\vec{A}_t = \{ A^{(1)}_t, A^{(2)}_t, \dots, A^{(N)}_t\}$
    \end{itemize}
\end{itemize}


\subsection{Joint Distribution}

\[
\begin{align}
    &p(S_1, S_2, \dots, S_T, A^{(1)}_1, A^{(2)}_1, \dots, A^{(N)}_1, A^{(1)}_2, \dots, A^{(N)}_T) \\ 
    &= p(S_1, S_2, \dots, S_T, \vec{A}_1, \vec{A}_2, \dots, \vec{A}_T) \\
    &= p(\vec{S}_{1:T}, A_{1:T}) \\
    &= p(S_1) \cdot \prod\limits_{t=1}^{T-1} p(S_{t+1} \mid S_t) \cdot \prod\limits_{n=1}^{N} \prod\limits_{t=1}^T p(A^{(n)}_t \mid S_t)
\end{align}
\]

\section{Factors}

Priors
\[
\begin{array}{c | c}
S_1 & p(S_1) \\ 
\hline
1 & \pi_1 \\ 
2 & \pi_2 \\ 
\vdots & \vdots \\
m & \pi_m \\ 
\end{array} 
\]

Transition
\[
\begin{array}{ccc}
\begin{array}{c c | c}
S_t & S_{t+1} & p(S_{t+1} \mid S_t) \\ 
\hline
1 & 1  & a_{1,1} \\ 
1 & 2  & a_{1,2} \\ 
\vdots & \vdots  & \vdots \\
1 & m  & a_{1,m} \\ 
2 & 1  & a_{2,1} \\ 
2 & 2  & a_{2,2} \\ 
\vdots & \vdots  & \vdots \\
m & m  & a_{m,m} \\ 
\end{array} 
&
\equiv
&
P^1 = 
\begin{bmatrix}
a_{1,1} & a_{1,2} & \dots & a_{1,m} \\
a_{2,1} & a_{2,2} & \dots & a_{2,m} \\
\vdots & \vdots & \ddots & \vdots \\
a_{m,1} & a_{m,2} & \dots & a_{m,m} \\
\end{bmatrix}
\end{array} 
\]


Emission
\[
\begin{array}{c c | c}
A^{(n)}_t & S_t & p(A^{(n)}_t \mid S_t) \\ 
\hline
1 & 1  & b_1^{(n)}(1) \\ 
1 & 2  & b_2^{(n)}(1) \\ 
\vdots & \vdots  & \vdots \\
1 & m  & b_m^{(n)}(1) \\ 
2 & 1  & b_1^{(n)}(2) \\ 
2 & 2  & b_2^{(n)}(2) \\ 
\vdots & \vdots  & \vdots \\
C_n & m  & b_m^{(n)}(C_n) \\ 
\end{array} 
\]


\section{EM Theory}

\begin{itemize}
    \item $\mathcal{H} = (S_t)_{t=1}^T$
    \item $\mathcal{D} = (\vec{A}_t)_{t=1}^T$ 
    \item $\Theta = (\vec{\theta}_1, \vec{\theta}_2, \vec{\theta}_3)$
    \item $\vec{\theta}_1 = \{\pi_1, \pi_2, \dots, \pi_m\} \equiv S_1 \text{ Priors}$
    \item $\vec{\theta}_2 = \{a_{1,1}, a_{1,2}, \dots, a_{m,m}\} = P^1 \equiv \text{Transition Probabilities}$
    \item $\vec{\theta}_3 = \{b_1^{(n)}(1), b_2^{(n)}(1), \dots, b_1^{(N)}(m)\} = P^1 \equiv \text{Emission Probabilities}$
\end{itemize}

\subsection{E-Step}

Hold $\Theta$ fixed and choose $q$ such that 
\[
    \begin{aligned}
        q(\mathcal{H}) &= p(\mathcal{H} \mid \mathcal{D}, \Theta) \\
        &= p(\vec{S}_{1:T} \mid A_{1:T}, \Theta)
    \end{aligned}
\]


\subsection{M-Step}

Hold $q$ fixed and optimise $\mathscr{L}(q, \Theta)$ w.r.t $\Theta$ 


After some math, this means finding $\Theta$ such that:
\[
    \begin{aligned}
        \Theta &= \underset{\Theta}{\operatorname{argmax}} Q(\Theta) \\
        &= \underset{\Theta}{\operatorname{argmax}} \sum\limits_{\mathcal{H}} q(\mathcal{H}) \cdot \text{log} \hspace{0.1cm} p(\mathcal{D}, \mathcal{H} \mid \Theta)
    \end{aligned}
\]




\newpage

\section{Update Equations}
Priors:
\begin{equation}
\pi_i^{\text{new}} = q(S_1=i)
\label{eq:prior_update}
\end{equation}

Transition Probabilities:
\begin{equation}
a_{ij}^{\text{new}} = \frac{\sum\limits_{t=1}^{T - 1} q(S_t=i,S_{t+1}=j)}{\sum\limits_{t=1}^{T-1} q(S_t=i)}
\label{eq:transition_update}
\end{equation}


Emission Probabilities:
\begin{equation}
b_i^{(n)}(j)^{\text{new}} = \frac{\sum\limits_{t=1}^T q(S_t = i) \cdot I(A_t^{(n)} = j)}{\sum\limits_{t=1}^T q(S_t = i)}
\label{eq:emission_update}
\end{equation}

We can now reference these equations by their label: Equation~\ref{eq:prior_update}, Equation~\ref{eq:transition_update} or Equation~\ref{eq:emission_update}. This is wicked, lemme tell you

\newpage

\section{FIGURES TIME}

things before figure



\section{Determining $m$}

Okay some questions:
\begin{enumerate}
    \item I have one sequence of $T$ observations with $N$ variables being observed at each $t$ step. Would my $k$ then be $T \times N$ or would it just be $T$?
    \item Define, algebraically, exactly what my $L$ should be that I need to calculate
\end{enumerate}

Okay progress update:
I have built the model. I have used synthetic data for model development. We have:


Note: Our attribute RVs are discrete. 

Model works with a set $m$ but the paper estimates this value using maximized log-likelihood, AIC and BIC
How do i do this? Is it just creating the model for various values of $m$ and choosing the one that returns the lowest AIC, BIC and largest log-likelihood?




 Consequently, in analyzing the drought indices, the number of latent states m was the first quantity we wanted to extract. We considered a simple procedure based on information criteria, which is a usual tool for model selection. The model with the optimal number of latent states was expected to best explain the data with a minimum number of free parameters. We used the  given as follows:
\[
    AIC = -2 \hspace{0.1cm} \text{log} L + 2p
\]

\[
    BIC = -2 \; \text{log}L + p \text{log} k
\]

Where $L \equiv$ the maximized value of the likelihood function for the estimated model

$p \equiv$ the number of free parameters, 

$k \equiv$ the number of data points. 


\subsection{How To Get $\text{log} \; \ell$}

I am not using the Forward Backward algorithm, I am using the EM algorithm that means the factors I have readily available are:

\begin{itemize}
\item $q(\mathcal{H}) = p(\vec{S}_{1:T} \mid A_{1:T}, \Theta)$
\item $p(S_1)$
\item $p(S_{t+1} \mid S_t)$
\item $p(A^{(n)}_t \mid S_t)$
\end{itemize}

If I can still use the forward equations, let me know. Otherwise I need to calculate the full joint distr and marginalise out?


A little bit embarrassing but how exactly do we get the log likelihood, the naive way. My understanding is that we:
\begin{enumerate}
    \item Calculate the full joint distribution: 
    \[
p(\vec{S}_{1:T}, A_{1:T}) = p(S_1) \cdot \prod\limits_{t=1}^{T-1} p(S_{t+1} \mid S_t) \cdot \prod\limits_{n=1}^{N} \prod\limits_{t=1}^T p(A^{(n)}_t \mid S_t)
    \]
    \item Marginalise out all $S_t$: 
        \[
p(A_{1:T} \mid \Theta) = \sum_{S_{1:T}} p(\vec{S}_{1:T}, A_{1:T} \mid \Theta)
        \]
    \item Then Observe Actual Data and sum the probs?
        \[
            \text{Likelihood} = \sum p(A_{1:T} = \mathcal{D}) ??
        \]
\end{enumerate}


I don't know what you mean by me being stuck. I get parameters which are the probabilities to the factors I am looking for. The model output is the max value $S_t$ which i get from my $q$ function. Let me know if I am overlooking something.


With regards to the AIC \& BIC calcs. you have here $\ell(\Theta) = \sum_{S_{1:T}} p(\vec{S}_{1:T}, A_{1:T})$ but this leaves us with a factor not a single value? This is required for AIC and BIC which are single values? This is why I thought you must sum over observations?




Okay Ill stop faffing. Forward-Backward is new, I didnt want to waste a time sink learning it. Especially because this LBU + EM is much more flexible of a route which is good for me since I plan to expand this model. One idea I have is to introduce second-order markov property to the thing. Is Forward-Backward still feasible for a DNBC with the second order markov property? If not I am sticking to LBU. And thus maybe need an alternative to AIC and BIC. But let me know your thoughts.


\newpage

\section{Questions}

\begin{enumerate}
    \item Forward-Backward Equations. I have to right? From what I can see it applies to second order as well when we vectorise our states? 
        \begin{itemize}
            \item Its only really a problem to try and get $log \; \ell(\Theta)$ for AIC and BIC. Besides this it works fine? 
            \item is extracting $S_t$ from $q(\mathcal{H})$ fine and correct? Since we want $p(S_t)$ not \\ 
                $p(S_t \mid A_{1:T})$...
        \end{itemize}
    \item Based on this as about LBU things:
        \begin{enumerate}
            \item emdw has this \verb|#include "lbu2_cg.hpp"|. What is this??
            \item Ask about LTRIP vs other methods $\rightarrow$ Other Methods: BETHE, JTREE
        \end{enumerate}
    \item Breaking symmetry for the priors $p(S_1)$. Is it necessary?
    \item With regards to BIC \& AIC, we need $k \equiv \text{Number of Free Parameters}$ (See calcs on \verb|model-dev-clean| pg 11)
    \item Discrete vs Cts Attribute RVs. See \verb|model-dev-clean| pg 10
    \item \verb|main.cpp| line 951

\end{enumerate}






