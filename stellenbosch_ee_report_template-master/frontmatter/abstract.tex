\chapter*{Abstract}
\addcontentsline{toc}{chapter}{Abstract}
\makeatletter\@mkboth{}{Abstract}\makeatother

\subsubsection*{English}
Drought remains a persistent challenge in South Africa, affecting agriculture, water security, and economic stability. Conventional monitoring often relies on single indices that capture limited aspects of drought behaviour. This study develops a probabilistic composite drought indicator using a dynamic naive Bayes classifier (DNBC) to integrate the standardised precipitation index (SPI), streamflow drought index (SDI), and normalised difference vegetation index (NDVI), representing meteorological, hydrological, and agricultural dimensions respectively.

The model, implemented for 1981--2019 in the southwestern Cape, estimated latent drought states using the expectation–maximisation (EM) algorithm with junction tree (JT) inference. Model selection, guided by standard criterion, identified a six-state configuration. Outputs were extracted via the Viterbi algorithm and Maximum Posterior Marginal (MPM) rule.

Results show the DNBC accurately identified major historical droughts and achieved the highest F1-score among inputs, demonstrating reliable classification and balanced sensitivity, supporting its use for probabilistic drought monitoring.

\selectlanguage{afrikaans}
\subsubsection*{Afrikaans}
Droogte bly ’n volgehoue uitdaging in Suid-Afrika en beïnvloed landbou, watersekerheid en ekonomiese stabiliteit. Konvensionele monitering steun dikwels op enkele indekse wat slegs beperkte aspekte van droogtegedrag vasvang. Hierdie studie ontwikkel ’n waarskynlikheidsgebaseerde, saamgestelde droogte-aanwyser deur gebruik te maak van ’n dinamiese naïewe Bayes-klassifiseerder (DNBC) wat die gestandaardiseerde reënvalindeks (SPI), stroomvloei-droogte-indeks (SDI) en genormaliseerde verskilmplantindeks (NDVI) integreer, wat onderskeidelik die meteorologiese, hidrologiese en landboukundige dimensies van droogte verteenwoordig.

Die model, geïmplementeer vir die periode 1981–2019 in die Suidwes-Kaap, het latente droogtetoestande geskat met behulp van die verwagting–maksimering (EM)-algoritme en die knoopboom (JT)-afleiding. Modelseleksie, gelei deur standaard kriteria, het ’n ses-toestand-konfigurasie geïdentifiseer. Uitsette is onttrek deur die Viterbi-algoritme en die Maksimum Posterior Marginale (MPM)-reël.

Resultate toon dat die DNBC groot historiese droogtes akkuraat geïdentifiseer het en die hoogste F1-telling tussen insette behaal het, wat betroubare klassifikasie en gebalanseerde sensitiwiteit aandui, en sodoende die gebruik daarvan vir waarskynlikheidsgebaseerde droogtemonitering ondersteun.

\selectlanguage{english}
