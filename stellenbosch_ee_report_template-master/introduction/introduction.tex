\graphicspath{{introduction/fig/}}
\chapter{Introduction}
\label{chap:introduction}
\section{Background}

\subsection{Drought as a Growing Threat}

Climate change is no longer a distant projection, but rather, is already reshaping how frequently and how severely extreme weather events occur. Recent reports and studies indicate that the frequency and intensity of droughts have markedly increased worldwide since the early 21st century. For instance, the OECD’s Global Drought Outlook reports that approximately 40\% of global land experienced upticks in both drought frequency and intensity when comparing the periods 1950--2000 to 2000--2020~\cite{Tyndall_2025}. Nature's ``Warming accelerates global drought severity'' highlights that, globally, drought magnitude has become more negative and that the number of drought months is increasing under observed climate conditions, whilst it is also being reported that multiyear droughts are becoming increasingly common~\cite{Gebrechorkos2025, Chen2025}.

\subsection{Water Demand, Vulnerability, and Regional Impact}

As global population continues to climb in South Africa at a rapid rate, water demand increases. Agriculture, industry and urban use all place stress on water systems, which are already under threat due to poor infrastructure and inequitable management. These two issues are prevalent in South Africa and only exacerbate the cost of drought~\cite{Olagunju19052019, Gebrechorkos2025}. Africa has been particularly vulnerable: since the 1960s, more than 382 drought events have affected millions of people, especially in Sahel and Southern Africa~\cite{SHIFERAW201467, Tyndall_2025}. In the Western Cape severe droughts have left lasting socio-economic scars with notable events including 1973--74, 1983--84, 1991--92, 1994--95, 2000--2001, 2003--2004, 2014--16, and 2017--18, each associated with sharp losses in crop yields, dam storages, and human hardship~\cite{Botai2017, socioeconomic_effects, BAUDOIN2017128, Sousa_2018}. 

The severe 1981--1984, multi-year drought across southern Africa demonstrated that water deficits in the region can be persistent and continent-scale. Recent climate analyses characterise the early 1980s event as among the most pronounced multi-annual rainfall deficits in the twentieth century for southern Africa. Consequences included widespread crop and livestock losses, major food-security interventions and sustained economic hardships in rural livelihoods that, in some catchments, persisted for several years after precipitation recovered. Such historical events are important because they illustrate not only acute system stress but also the long tail of socio-economic recovery following protracted drought.

A second, and more recent episode is the 2015--2018 drought in the Western Cape which revealed multiple systemic vulnerabilities in both infrastructure and governance. The region experienced severe municipal restrictions as reservoir storages declined to between roughly 15--30\% of capacity, provoking near-municipal “Day Zero” scenarios, emergency demand management and extraordinary conservation measures. The drought also produced substantial agricultural economic losses, associated labour reductions, and marked pressures on public-health and social services~\cite{socioeconomic_effects,drivers_of_drought_som,Sousa_2018}. 

The crisis in the Western Cape also exposed the limits of urban water supply designs that assume relatively steady inter-annual availability, and it highlighted institutional gaps in reservoir operation, intergovernmental coordination and demand-side planning. Analyses of the City of Cape Town response emphasise how communications, behavioural change and temporary policy levers averted the most catastrophic outcomes, but also that these were last-resort measures that imposed disproportionate burdens on low-income communities and agricultural producers dependent on the urban market. Reports and post-event reviews point to the need for improved system modelling, diversified supply portfolios and explicit drought contingency plans at municipal and provincial levels~\cite{Joubert_Ziervogel_2019,Babajide}.

Drought has direct consequences for agricultural productivity, human and animal health, and vegetation cover, with water scarcity leading to food insecurity and poverty~\cite{SHIFERAW201467}. Indirectly, drought can contribute to environmental degradation, exacerbate food shortages, diminish human welfare, and, in certain contexts, act as a catalyst for social unrest~\cite{https://doi.org/10.1155/2014/508953}. Across Africa, the agricultural sector has borne significant impacts, manifesting as the degradation of grazing lands, crop failure, depletion of farming assets, and the impoverishment of farmers, particularly vulnerable smallholder farmers, often culminating in forced migration from rural to urban areas~\cite{SHIFERAW201467}. 

South Africa’s recent and historical droughts make clear that water scarcity is a clear risk that is worsened by poor infrastructure, governance constraints and socio-economic inequality. This points to the need for more integrated monitoring and decision-support tools.

\subsection{Complexity of Drought}
Not only are the impacts of drought multifaceted, but drought itself is a complex and multifaceted phenomenon that resists a simple or universal definition~\cite{Llyod+Benjamin}. Unlike discrete natural disasters such as floods or earthquakes, drought unfolds gradually, often with indistinct onset and termination periods. This complexity arises from the fact that drought is not merely a physical phenomenon but a convergence of meteorological, hydrological, agricultural, and socio-economic processes, as defined by Wilhite and Glantz~\cite{Wilhite+Glantz}. Consequently, researchers and policymakers have approached the study and monitoring of drought through a wide range of indices, each of which seeks to capture one particular dimension of this broader phenomenon.

Let us now look at a brief explanation of each category. \textit{Meteorological drought} is defined as a period of significantly below-average precipitation, which typically serves as the primary trigger for drought conditions and is often quantified by indices such as the standardised precipitation index (SPI) or the standardised precipitation-evapotranspiration index (SPEI). These indices compare current precipitation levels to long-term historical averages for a specific region~\cite{spi_seminal_paper, douville2021water, spei_seminal_paper, wmo_indice_guide}. 

However, such meteorological measures alone cannot capture subsequent and cumulative effects on hydrological systems, ecosystems and human livelihoods. \textit{Hydrological drought} describes reductions in surface and subsurface water resources, such as streamflow, groundwater tables and reservoir storage. This type of drought typically lags behind meteorological drought and is measured using indices such as the streamflow drought index (SDI) or the standardised streamflow index (SSI), which are metrics derived from river monitoring~\cite{sdi_seminal_paper, Loon+Anne, wmo_indice_guide,ssi_seminal_paper}. 

\textit{Agricultural drought} describes the phenomenon where the climate causes a significant decline in crop yield or quality. Consequently, its measurement focuses on soil moisture availability, crop yield, and vegetation health. The latter is increasingly quantified using remote sensing indices like the normalised difference vegetation index (NDVI)~\cite{judith2025remote}. Another common index to use for this aspect of drought is the evaporative stress index (ESI) which quantifies anomalies in evapotranspiration. It is important to note that agricultural drought is a broader concept than purely meteorological drought, as it can be induced or exacerbated by non-environmental factors. However, these socio-economic factors, such as inadequate irrigation infrastructure or poor land management practices, often determine the severity of the impact that a precipitation deficit has on agricultural output~\cite{ndvi_seminal_paper, Maracchi2000, wmo_indice_guide, esi_seminal_paper}.

\textit{Socio‐economic drought} encompasses the human consequences of water scarcity and agricultural failure: it occurs when demand for water, food or energy exceeds supply due to drought disruptions, manifesting in outcomes such as food insecurity, income loss, migration or social unrest~\cite{w17071002}. Although socio‐economic drought is difficult to quantify directly, researchers have attempted to capture it via composite indices integrating the three types of drought mentioned above. Additionally, they have experimented with vulnerability and economic or social indicators to measure human exposure and impacts~\cite{WANG2022131248,Mehran+etAll, wmo_indice_guide}.

To make matters worse, these different facets of drought manifest differently across South Africa’s varying climate zones. The Western Cape sees winter rainfall with a Mediterranean climate, the East Coast sees summer rainfall and a subtropical climate, while the interior regions of the country are semi-arid. This spatial heterogeneity alters the timing, lag and propagation of drought~\cite{za_drought_review2, mulenga}.

Indices designed for a single disciplinary perspective (meteorological, hydrological or agricultural) will emphasise different events and different timings. This creates conflicting information from each index, which complicates interpretation and leads to poor decisions. In a country with contrasting rainfall regimes, this means that a single index cannot reliably capture exposure, vulnerability and impact across all regions. This is a core reason to pursue integrated or composite monitoring approaches~\cite{Drought.gov}. 


\subsection{Towards Integrated Drought Monitoring in South Africa}
Conventional drought indices each capture a particular physical or ecological dimension of drought, namely, the SPI for meteorological drought, SDI for hydrological drought, and NDVI for agricultural drought. Relying on any single index therefore provides an incomplete view. These indices frequently contradict each other requiring industry experts to analyse them, ultimately leading to false positives and negatives for different users. This complicates decision-making when policymakers require a consistent, interpretable drought declaration~\cite{NCEI_2024}.

A composite indicator aims to combine the output of different, well-established indices to gain a more holistic assessment of drought exposure and its impacts. The benefits include improved detection of drought impacts, more robust signals through redundancy across inputs, and clearer communication to stakeholders who require an integrated risk of drought. Composite models such as the U.S. Drought Monitor and the European Combined Drought Indicator demonstrate how convergent evidence can be used to perform weekly or monthly monitoring. It should be noted that composite approaches are not plug-and-play; they require careful design choices and are sensitive to input quality~\cite{usdm,cdi,some_comp_indicator}.

South Africa has made progress in index development and in the use of multiple indices, but the literature and operational practice still lack a widely-adopted, national composite drought product akin to the United States drought monitor (USDM). Recent reviews of drought monitoring in southern Africa highlight that integrated, multivariate approaches are increasingly recommended, however, composite indices in a South African context remain scarce~\cite{za_drought_review, za_drought_review2}. 


\subsection{Probabilistic Graphical Models and Their Value in Drought Monitoring}
Probabilistic graphical models (PGMs) constitute a family of statistical models that combine principles from probability theory and graph theory to represent complex systems of interdependent variables. These models enable principled learning and inference under uncertainty by encoding joint probability distributions in a graphical form~\cite{koller_textbook}.

PGMs are powerful tools for creating composite drought indicators as they are able to handle the field's inherent uncertainties which include data sparsity, measurement errors, and complex climate relationships. Other notable approaches typically fall into two categories. The first involves \textit{weighted aggregation}, where normalised drought indices are combined using fixed or subjectively assigned weights \cite{comp_drought_method_linear}. This method is simple, assumes independence between indices, and imposes a static linear relationship. This does a poor job at capturing the complexities of drought. The second approach employs \textit{dimensionality reduction techniques}, with principal component analysis (PCA) being the most widely used~\cite{atmos16070818}. PCA assumes linear relationships between variables, is sensitive to both outliers and scaling, and has poor interpretability. PGMs on the other hand offers a principled and an interpretable means of modelling both the dependencies and uncertainties of climatic conditions. When applied to the construction of composite indicators, PGMs can explicitly encode how different aspects of drought influence one another over time. Models like the dynamic naive Bayes classifier (DNBC), a variant of the hidden Markov model (HMM), extend this capacity by incorporating temporal dependencies between states. This enables the model to infer drought evolution using sequential data. Additionally, PGMs produce probabilistic outputs that quantify the likelihood or confidence of different drought classifications---something that can not be done using the other two methods mentioned.

Recent studies have been successfully incorporating DNBCs in other countries, most notably in South Korea, to combine individual indices into an integrated multiple-drought index. These studies showed improved detection through the output of their probabilistic models compared with single indices alone. They illustrate the technical feasibility of the DNBC approach and provide a methodological blueprint for adapting such a classifier to a South African context. Crucially, however, the transfer of these methods to South Africa requires careful calibration to local climates, and of course, data availability~\cite{dnbc_drought_second, dnbc_drought_first}.

\section{Problem Statement}
South Africa lacks a composite drought indicator that integrates the meteorological, hydrological, and agricultural dimensions of drought. Existing systems tend to focus on individual indices which capture isolated aspects of drought but fail to represent the full complexity of the problem. As a result, decision-makers lack a cohesive view of drought and thus struggle to implement timely and effective intervention strategies.

The problem addressed in this study is therefore the absence of a probabilistically principled framework capable of combining heterogeneous drought indicators into a single, adaptive, and interpretable composite index. The desired outcome is a model that integrates three indices, that being the SPI, SDI, and NDVI, into a probabilistic structure capable of inferring drought state transitions over time. 

The scope of this study is deliberately restricted to the meteorological, hydrological, and agricultural domains of drought. While the socio-economic dimension is important, it is excluded. This exclusion is primarily due to the considerable complexity involved in quantifying socio-economic indicators in the South African context. For existing methods, the limited availability of reliable, open-access data makes it near impossible. The model is applied to the southwestern Cape region of the country using openly available datasets covering the period 1981--2019. 

Finally, it will also be noted that, fundamentally, the DNBC itself has several naive assumptions which will be discussed later. Although these assumptions are naive, the purpose of this study is to evaluate the effectiveness of this framework as a drought monitoring tool in the South African context.

\section{Overview of Project Design}
\label{sec:overview_design}
% Intro
This section provides a high-level overview of the design and methodology underpinning this project. The primary goal is to develop a probabilistic drought monitoring framework for South Africa based on a DNBC. The following subsections outline the complete workflow---from data acquisition and index derivation to model construction, state interpretation, and performance evaluation. The intention here is not to provide full theoretical or mathematical detail, but to clarify the logical sequence through which the final drought classification system was built and assessed.

% Data Acquisition and Preprocessing 
The study spans the period 1981--2019 and focuses on areas in the southwestern Cape region (locations of data sources are plotted in Figure~\ref{fig:study-area}). All datasets used are open-source. Precipitation and river streamflow observations are collected, cleaned, and processed to compute the SPI and the SDI respectively. The SPI quantifies the short-term rainfall anomalies captured at weather stations while the SDI measures streamflow deficits at river gauging stations. Both of these are commonly used as statistical measures to capture the meteorological and hydrological aspects of drought respectively. In contrast, NDVI captures vegetation conditions, which serves as a proxy for agricultural drought. This data is obtained directly from the National Oceanic and Atmospheric Administration (NOAA) who maintain a global data record of NDVI across the globe~\cite{ndvi_data}. These three indices were discretised and formatted to form the input for the DNBC.

% Model Structure and Integration
The DNBC represents the evolution of drought conditions by linking the input indices (SPI, SDI, and NDVI) to a hidden or latent discrete RV at each time step. This structure allows the latent RVs to capture the true, underlying drought dynamics by integrating the different dimensions of each input index while also accounting for temporal dependencies. 

Model training is performed using the expectation-maximisation (EM) algorithm, which iteratively estimates the model parameters that maximise the likelihood of the observed data (i.e., the input indices). Inference over the probabilistic structure is implemented via the junction tree (JT) algorithm for efficient computation of joint probabilities.

The number of hidden drought states (i.e., the cardinality of the latent RV) is not fixed but rather determined through model selection, alongside the time scales (rolling window size) of both the SPI and SDI. Models with these varying hyperparameters are evaluated using the Akaike information criterion (AIC), Bayesian information criterion (BIC), and the maximised log-likelihood.

Once the optimal hyperparameters are identified, the model is then re-trained with this final structure. This yields, for each time step, a probability distribution over a finite set of discrete drought states. These states can be interpreted along a conceptual scale from dry to wet conditions, where the binned values below the midpoint represent progressively drier states and those above it progressively wetter states. Thus, rather than assigning a single categorical label, the model quantifies the severity of drought at each time step. Figure~\ref{fig:comp-synthesis} shows a diagram illustrating how the input and output components work with the DNBC model where the number of latent drought states is $6$ as an example.
\begin{figure}[!h]
    \centering
    \includegraphics[width=\linewidth]{skripsie-component-synthesis.png}
    \caption[Overview of Inputs and Outputs of the DNBC]{Schematic of the end-to-end project pipeline, showing data preprocessing, index calculation, DNBC input formatting, and final classification into six latent drought states as an example.}
    \label{fig:comp-synthesis}
\end{figure}

In order to extract the categorical drought classifications, the DNBC is decoded using two complementary approaches. The first is the Viterbi algorithm, which identifies the single most probable sequence of hidden drought states over the entire period, providing a coherent narrative of the drought's evolution. The second is the maximum posterior marginal (MPM) rule, which determines the most likely drought state for each individual time step, along with an associated probability that serves as a measure of uncertainty for the classification. Using both methods provides a more robust and nuanced classification of drought while being interpretable.

% Evaluation and Summary
Finally, model performance is evaluated against the individual input indices (SPI, SDI, and NDVI) to assess whether the DNBC provides a more robust and cohesive representation of drought. Evaluation focuses on the model’s ability to correctly identify major historical drought events in the study area (1983--1984, 1991--1992, 1994--1995, 2000--2001, 2003--2004, 2014--2016, and 2017--2018). Both quantitative and qualitative analyses are conducted. Temporal plots are used to visually compare drought classifications across indices, while statistical metrics---including recall, precision, F1-score, and overall accuracy---are applied to objectively quantify performance.

% Concluding paragraph
This overview summarises the end-to-end design of the project, from data source to probabilistic classification. Later chapters will provide more clarity and detail regarding the theory, implementation, and results. For a visual representation of this design, see Appendix~\ref{app:project-design-diagram}.

\section{Contributions}
This project contributes to drought monitoring research in three distinct ways. 

Firstly, it introduces a probabilistic, data-driven framework for composite drought monitoring in South Africa. This offers an alternative to other conventional methods which often lack interpretability. By integrating SPI, SDI, and NDVI the study demonstrates that PGMs can effectively capture cross-domain and temporal dependencies in drought evolution.

Secondly, it evaluates the DNBC's drought monitoring capability by comparing its detection of major historical droughts against trusted industry indices. This provides practical evidence for the model's operational feasibility.

Finally, this project contributes a framework for composite drought indicator development in South Africa using only open-source data while keeping reproducibility in mind. This work sets the stage for future early-warning systems that can adapt to uncertainty and provides a clear method for adding other indicators, like socio-economic data, later on.

In doing so, the study not only fills a critical gap in South African drought monitoring but also provides methodologies for composite indicator development in data-sparse environments.

\section{Report Outline}
This report is organised into four main sections, each addressing a key stage in the development and evaluation of the proposed composite drought indicator.

\textbf{Chapter~\ref{chap:literature}: Literature Review and Theoretical Framework} \\
This chapter reviews existing approaches to composite drought monitoring and establishes the theoretical foundation for the modelling framework used in this study. The first part surveys prior research on multi‐index integration, drought index evaluation for South African conditions, and the application of dimensionality-reduction techniques. The second part introduces PGMs, describing their structure, inference mechanisms, and learning algorithms.

\textbf{Chapter~\ref{chap:methods}: Methodology} \\
Chapter~\ref{chap:methods} details the full project design pipeline, from data acquisition to model implementation. It begins with a description of the data sources, preprocessing steps, and the computation of the individual drought indices. This is followed by a detailed exposition of the DNBC model design, including its structure, inference procedures, parameter estimation, and model selection criteria. The chapter concludes by outlining the computational environment and the implementation of the end-to-end data and modelling workflow.

\textbf{Chapter~\ref{chap:results}: Results} \\
This chapter presents the outcomes of model training, evaluation, and analysis. It first discusses model selection and the definition of drought states, followed by an examination of model behaviour through latent-state sequences and confidence analysis. Quantitative evaluation metrics, including accuracy, precision, recall, and F1-score, are then used to assess model performance relative to the individual drought indices. 

\textbf{Chapter~\ref{chap:conclusion}: Summary and Conclusion} \\
The final chapter summarises the main findings of the study and evaluates the extent to which the research objectives were achieved. It reflects on the methodological and practical contributions of the DNBC-based framework to drought monitoring and concludes by identifying limitations and proposing directions for future research. 
