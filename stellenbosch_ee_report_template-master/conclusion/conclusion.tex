\graphicspath{{conclusion/fig/}}

\chapter{Summary and Conclusion}
\label{chap:conclusion}

This project sought to develop and evaluate a principled, probabilistic approach for drought monitoring in the South African context using a dynamic Naive bayes classifier (DNBC). The model combines three key drought indicators: the standardised precipitation index (SPI), streamflow drought index (SDI), and normalised difference vegetation index (NDVI), to construct a composite drought indicator that captures the complexity of meteorological, hydrological, and agricultural dimensions of drought. 

The approach was implemented over the period of 1981--2019 with the study area being in the southwestern Cape region using open source datasets. Raw precipitation and streamflow data were collected, cleaned and formatted for index calculation. In contrast, the NDVI was obtained directly from a dataset provided by the National Oceanic and Atmospheric Administration (NOAA). These three indices were then discretised which together formed the input dataset for the DNBC model. The model was formulated with discrete random variables (RVs) representing latent drought states and observed input indices. Parameter estimation was performed using the expectation-maximisation (EM) algorithm paired with the junction tree (JT) algorithm for inference, and model selection was guided by Akaike information criterion (AIC), Bayesian information criterion (BIC), and maximised log-likelihood criteria.

The model was evaluated both qualitatively and quantitatively, with plots revealing that the DNBC successfully identified known drought periods in the Western Cape. The resulting DNBC was reliable when it predicted drought, but missed a number of true drought events. Nonetheless, the DNBC’s performance was found to be comparable to, and in some respects better than, the individual indices as it achieved the highest F1-score among all evaluated methods. This suggests that the composite indicator successfully captured abstract information across the different drought dimensions.

\section*{Future Work and Recommendations}

Although the DNBC demonstrated promising results, several paths exist for further improvement and exploration:

\begin{itemize}
    \item \textbf{Continuous Inputs:} This implementation discretised all input indices. Extending the DNBC to handle continuous RVs, for example via Gaussian or hybrid emission distributions, could increase the model's capacity to capture underlying drought dynamics.
    \item \textbf{Expanded Input Set:} Future models could integrate additional indices such as soil moisture, evapotranspiration, or temperature-based indicators to better capture multi-dimensional drought processes.
    \item \textbf{Alternative Methods:} Exploring other probabilistic and machine learning approaches such as random forests, support vector machines or even deep learning methods may have a greater capacity to capture the complexity of drought. 
\end{itemize}

\noindent In summary, this work demonstrates that probabilistic graphical models, specifically the dynamic naive Bayes classifier, represent a viable approach to drought characterisation in data-limited environments. While challenges remain, the results show that integrating multiple drought dimensions using a principled and probabilistic approach yields both interpretive and operational value. With further refinement this approach could form the foundation of a robust drought monitoring system for South Africa.

