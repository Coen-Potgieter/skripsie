\graphicspath{{conclusion/fig/}}

\chapter{Summary and Conclusion}
\label{chap:conclusion}

This project sought to develop and evaluate a principled, probabilistic approach for drought monitoring in the South African context using a Dynamic Naive Bayes Classifier (DNBC). The model combines three key drought indicators, the Standardised Precipitation Index (SPI), Streamflow Drought Index (SDI), and Normalised Difference Vegetation Index (NDVI), to construct a composite drought indicator that captures the complexity of meteorological, hydrological, and agricultural dimensions of drought. 

The approach was implemented over the period of 1981–2019 with the study area being the Western Cape using open source datasets. The DNBC was formulated with discrete random variables representing latent drought states and observed input indices. Parameter estimation was performed using the Expectation–Maximisation (EM) algorithm paired with Junction Tree (JT) inference, and model selection was guided by AIC, BIC, and maximised log-likelihood criteria.

The model was evaluated both qualitatively and quantitatively, with plots revealing that the DNBC successfully identified the known drought periods in South Africa. However, not only did the model output exhibit oscillatory behaviour within relatively short time periods, showing a false representation of actual drought dynamics, it also produced many false alarms for its classifications. Nonetheless, the DNBC’s performance was found to be comparable to, and in some respects better than, the individual indices as it achieved the highest F1-score among all evaluated methods. This suggests that the composite indicator captures abstract information across the different drought dimensions.

\section*{Reflection on Objectives}
The objectives set out at the beginning of this work were to develop a composite drought indicator using a DNBC, to evaluate its performance relative to established indices, and to assess its applicability within the South African context. Each of these objectives was achieved. The DNBC framework was successfully designed and implemented, its performance compared against established drought indices, and its results were obtained despite the unique data and climatic challenges of South Africa. Overall, the findings indicate that the DNBC-based composite indicator performs at least comparably to existing indices, and arguably better.

\section*{Future Work and Recommendations}

Although the DNBC demonstrated promising results, several paths exist for further improvement and exploration:

\begin{itemize}
    \item \textbf{Continuous Inputs:} This implementation discretised all input indices. Extending the DNBC to handle continuous random variables, for example via Gaussian or hybrid emission distributions, could preserve more information and reduce output variability.
    \item \textbf{Enhanced Data Quality:} Data reliability and consistency remain a significant limitation in the South African context. Access to higher quality rainfall, streamflow, and remote-sensing data would likely improve both the model’s calibration and generalisation.
    \item \textbf{Expanded Input Set:} Future models could integrate additional indices such as soil moisture, evapotranspiration, or temperature-based indicators to better capture multi-dimensional drought processes.
    \item \textbf{Alternative Methods:} Exploring other probabilistic and machine learning approaches such as Random Forests, Support Vector Machines or even deep learning methods may have a greater capacity to capture the complexity of drought. 
\end{itemize}

\noindent In summary, this work demonstrates that probabilistic graphical models, specifically the DNBC, represent a viable approach to drought characterisation in data-limited environments. While challenges remain, particularly regarding data availability and noise in model output, the results show that integrating multiple drought dimensions using a principled and probabilistic approach yields both interpretive and operational value. With further refinement and expanded datasets, this approach could form the foundation of a robust drought monitoring system for South Africa.



