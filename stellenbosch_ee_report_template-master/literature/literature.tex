\graphicspath{{literature/fig/}}

\chapter{Literature Review}
\label{chap:literature}

\subsection*{Introduction}

An attempt at advancing drought monitoring depends on a substantial foundation of prior research. Scholars have investigated methods ranging from traditional single-index approaches to more sophisticated probabilistic models. Notably, South Korean research has successfully employed Dynamic Naïve Bayes Classifiers (DNBCs) to develop composite drought indicators, and Hidden Markov Models (HMMs) have been used elsewhere to model drought dynamics. In contrast, the majority of South African studies concentrate on single indices for regional analyses, paying little attention to composite indicators. This review synthesises key contributions from these research areas to provide necessary context before continuing.

\subsection*{Development of a Multiple-Drought Index for Comprehensive Drought Risk Assessment Using a Dynamic Naive Bayesian Classifier}

In this study the authors developed a Dynamic Naive Bayesian Classifier multiple-drought index (DNBC-MDI) to produce a probabilistic, multi-dimensional assessment of drought risk. Their stated objectives were to combine conventional drought indices (SPI, SDI, ESI and WSCI) using a DNBC, to apply the resulting DNBC-MDI to bivariate drought-frequency analysis for risk estimation, and to investigate future changes in drought risk under an RCP8.5 climate scenario. Methodologically, the study focused on the Han River basin and used observed data for 1974–2016 together with synthetic climate projections for 2017–2099 generated by the HadGEM2-AO model under RCP8.5 scenario. The DNBC parameters were estimated with an expectation–maximisation (EM) algorithm using the \texttt{depmixS4} package from R. Bivariate drought frequency was assessed using a Clayton copula, and a risk equation was employed to compute 100-year return-period risks. The principal results showed that the DNBC-MDI achieved the highest average classification accuracy compared with the individual indices, whilst successfully reproducing several known drought episodes (1994–1995, 2001, 2008–2009, 2012, 2014–2015). The authors were very candid about their limitations, which are as follows: 
\begin{enumerate}
    \item The assessment focused predominantly on climate model outputs, disregarding remote-sensing products. For this they suggest using MODIS or Landsat.
    \item The model assumes conditional independence among the input indices. This assumption is brittle given the interconnections between precipitation, streamflow and evapotranspiration processes. 
\end{enumerate}
Overall, the paper demonstrates the technical feasibility and potential advantages of a DNBC-based composite indicator for drought characterisation. Simultaneously, it also signals important areas of concern with regards to robustness and transferability for an adaptation~\cite{dnbc_drought_second}.

\subsection*{Assessment of Probabilistic Multi-Index Drought Using a Dynamic Naive Bayesian Classifier}

This paper wanted to apply a DNBC to integrate multiple drought indices into a single, coherent drought state representation. The objectives were to combine indicators from different feature spaces, that being: SPI for meteorological, SDI for hydrological, and NVSWI for agricultural. Additionally, they wanted to evaluate whether the DNBC-based drought states could outperform individual indices in terms of detection, classification, and persistence. The study was carried out in the Han River upstream sub-basin in South Korea, using data from 1980–2015 for in-situ observations and 2003–2015 for MODIS-derived indices. The DNBC was constructed with five hidden drought states, the number selected using AIC, BIC and minimum log-likelihood for model selection criteria, and parameters were estimated through the EM algorithm implemented in the \texttt{depmixS4} R package.

The key results showed that the DNBC-based drought states successfully reproduced known drought episodes (2004, 2006, 2008–2009, 2014, 2015) and provided accurate representations of drought duration and persistence. In detection performance, DNBC-DS captured 100\%, 96\%, 100\%, and 93\% of droughts identified by SPI, SDI, NVSWI, and a composite drought index (CDI) respectively. The approach also highlighted the differing relationships between indicators, with strong correlation between SPI and SDI with a score of 0.648, but weak correlations involving NVSWI (0.186–0.187). Overall, the DNBC offered a probabilistic framework for drought monitoring that explicitly incorporated uncertainty, outperforming deterministic single-index approaches. 

Regardless, the authors acknowledged some of their key limitations. Firstly, the model relied on only three indices, which is not complex enough to capture what we call drought. It excluded potentially informative variables such as temperature, water vapour, and radiation. Beyond these, aligning with the paper above, this model also assumes that the input indices are conditionally independent once again making a brittle assumption.

Despite these constraints, the study offered a structured path toward a more holistic multi-indicator integration and contributed to the validity of using DNBCs for composite drought indicators~\cite{dnbc_drought_first}. 

\subsection*{Review of In-Situ and Remote Sensing-Based Indices and Their Applicability for Integrated Drought Monitoring in South Africa}

This study aimed to critically assess the performance and applicability of both in-situ and remote sensing-based drought indices for integrated drought monitoring in South Africa. Its objectives were to evaluate eight widely used indices and to determine which are most suitable for South Africa’s highly variable climate. These eight indices were: PDSI, SWSI, VCI, SPI, SPEI, SSI, SGI, and GRACE-based indices. A further aim was to test the hypothesis that no single index can adequately capture all aspects of meteorological, agricultural, and hydrological drought. 

They followed the World Meteorological Organisation’s (WMO) 2016 guidelines for drought indicator assessment. They used five evaluation criteria focusing on capability, sensitivity, data requirements, computational simplicity, and versatility for integration. The review drew from published studies in South Africa and other regions with similar climate characteristics. The indices were chosen based on surveys, while their feasibility was assessed against the evaluation framework mentioned. 

The findings demonstrated that the PDSI and SWSI are not feasible to obtain in South Africa due to their high complexity with regards to data requirements. However, SPI, SPEI, VCI, SSI, and SGI were identified as the most feasible candidates for integrated drought monitoring because of their simplicity and adaptability. Regardless, calculation issues remain, for example, there is no consensus on the most suitable probability distribution functions (PDF) for the calculations of SSI and SGI, with the most commonly used Gamma distribution performing poorly in South African catchments. Some alternative distributions showed improved results but inconsistencies persisted. Finally, the review recommended exploring multivariate approaches that combine SPI, SPEI, VCI, SSI, and SGI, while also noting the potential of GRACE-based indices, particularly with regards to groundwater, in order to compensate for the country’s limited groundwater records. 

The study transparently noted some important limitations. Data availability constraints undermine the feasibility of effective indices such as the PDSI and SGI, while the scarcity, or absence, of groundwater records limits applications. PDF selection for SSI and SGI remains uncertain given the climate variation in South Africa. The authors identified key research gaps within the nation, including the need for multivariate index testing and more exploration of GRACE-based products.

Ultimately, the review emphasised that integrated approaches, underpinned by sensitivity analysis and comparative testing, are required to strengthen drought monitoring in South Africa’s complex climatic landscape~\cite{za_drought_review}.

\subsection*{Developing a Composite Drought Indicator Using PCA Integration of CHIRPS Rainfall, Temperature, and Vegetation Health Products for Agricultural Drought Monitoring in New Mexico}

The objective of this study was to construct a Composite Drought Indicator for New Mexico, the so called CDI-NM, by integrating multiple variables through Principal Component Analysis (PCA). The research sought to provide a drought monitoring tool capable of identifying historical drought events, while also quantifying drought extent across the state. The study combined satellite-derived rainfall, temperature, and vegetation health products to demonstrate the effectiveness of PCA and to investigate drought impacts on agricultural production. 

The methodology focused on New Mexico which is an agriculturally important US state and is vulnerable to varying climates. Four input datasets spanning 2003–2019 were incorporated: CHIRPS rainfall data, MODIS Land Surface Temperature (LST), Smoothed Normalized Difference Vegetation Index (SMN), and Vegetation Condition Index (VCI). PCA was conducted independently for each month, with suitability being validated using Kaiser-Meyer-Olkin and Bartlett’s tests. They tested their model output by comparing it against SPI-3 and by correlating it with the annual variations in the yields of wheat, corn, peanuts, and cotton.

The results indicated that CDI-NM showed strong agreement with SPI-3, effectively capturing major drought events in 2003, 2011–2013, and 2018. Additionally, the showed their CDI-NM had strongly negative correlations with yields for corn (-0.68) and wheat (-0.63), while having a weaker correlation with cotton (-0.20). This reflects greater drought tolerance for cotton. Relationships between input variables were also consistent with expectation, as positive correlation was seen between VCI and rainfall (0.78) and negative correlation with LST (-0.43). Finally, the indicator demonstrated more natural variations than SPI, suggesting improved ability at capturing agricultural drought. 

Despite these achievements, several limitations were identified. The method of PCA relies on linear assumptions, temporal stationarity, and is sensitivity to scaling. The 17-year dataset is inherently limited with regards to long-term generalisability. Some data-related uncertainties further constrained precision. Redundancy between NDVI-derived SMN and VCI also posed risks of over-representation of vegetation conditions. Moreover, the study did not conduct sensitivity testing of PCA-derived weights, leaving gaps in applicability. The authors highlighted the need for longer datasets, uncertainty assessments, and more advanced dimensionality reduction techniques to strengthen the reliability of composite indicators for drought monitoring~\cite{atmos16070818}.

\subsection*{Conclusion}
The literature shows that probabilistic approaches are promising for capturing drought's complex nature, offering an advantage over traditional indices. Although South Korean research offers a strong blueprint, the scarcity of composite indicator development in South Africa reveals a significant research gap. This project seeks to bridge this gap by tailoring a DNBC to South Africa. 
