\graphicspath{{methods/fig/}}

% TODO: DO THIS: Add implementation here as well like discretisation or whatever the fuck
\chapter{Methods}
\label{chap:methods}
\section{Data Acquisition}
\label{sec:data_acquisition}
The development of a composite drought indicator requires careful selection of input variables that capture the different aspects of drought. Three indices were selected: the Standardised Precipitation Index (SPI) to represent meteorological drought, the Streamflow Drought Index (SDI) to represent hydrological drought, and the Normalised Difference Vegetation Index (NDVI) as a proxy for agricultural drought. These indices were chosen based on their widespread use in literature and the availability of data. Data scarcity is a challenge in South Africa, as openly accessible, long and consistent drought-related datasets are limited. Consequently, the choice of indices attempts to strike a balance between theory and pragmatic constraints~\cite{za_drought_review,za_drought_review2,dnbc_drought_first,dnbc_drought_second}.

Fix only the last sentence, leave everyhting else as is
\subsection{Sources}
\label{sec:data_sources}
To compute the SPI, monthly rainfall data was obtained from the University of Cape Town (UCT) dataset, which covers the period 1979–2019 (Dataset~\cite{uct_data}). The dataset provides rainfall values at station level. This offers a high degree of granularity across South Africa.

For the SDI, daily streamflow records were obtained from the Department of Water and Sanitation (DWS), which maintains audited historic data regarding hydrology (Dataset~\cite{DWS_2011}). These daily records were averaged to create monthly records, which was then used to calculate the desired index.

To obtain the NDVI, the NOAA Climate Data Record (CDR) of AVHRR Normalised Difference Vegetation Index (NDVI), Version~5 was used (Dataset~\cite{ndvi_data}). The dataset spans the period 1981–2025 and is provided in global NetCDF format. For the purposes of this study, only the South African subset was extracted. This required targeted downloading and filtering, given the large size of the global dataset.

It is important to note that the overlapping period of the available data, and thus the scope for the model output spans from 1981 to 2019.

\subsection{Preprocessing}
To prepare the indices for model input, several preprocessing steps were performed:  
\begin{enumerate}
    \item \textbf{Time Period Alignment:} All datasets were resampled or aggregated to a common monthly resolution.  
    \item \textbf{Area Alignment:} For station-based datasets, like rainfall and streamflow, records were harmonised by selecting stations with consistent temporal coverage. For NDVI, gridded data was averaged over the area of choice. 
    \item \textbf{Brief Exploration Of Data:} The data sets were analysed to identify and issues in the data such as missing/null values, format consistency, validity, etc. No problems were found
\end{enumerate}


\section{Index Calculation}
\label{sec:index-calc}
The collected data was transformed into drought indices using well-known methodologies. As mentioned, each index captures different dimensions of drought conditions. Together they provide a more comprehensive view than any single measure. \\
Below is a brief overview of each index and its mathematical formulation. Full derivations and discussions are available in the cited references.

\subsection{Standardised Precipitation Index (SPI)}

The SPI is based solely on precipitation and measures anomalies relative to the long-term probability distribution at a given location and timescale. The calculation involves fitting a long-term precipitation record to a gamma distribution, which is then transformed into a standard normal distribution~\cite{spi_seminal_paper}.

The SPI can be computed for different time scales (e.g., 1, 3, 6, or 12 months). This study employs the 3-month timescale (SPI-3), which is widely used for monitoring. Negative SPI values indicate dry conditions, with established thresholds categorising drought severity.

\subsection{Streamflow Drought Index (SDI)}

The SDI extends the framework of the SPI to characterise hydrological drought using streamflow volumes. Monthly streamflow observations are aggregated and standardized in a similar manner to the SPI calculation~\cite{sdi_seminal_paper}.

This method allows for the identification of periods with below-normal streamflow, indicating hydrological drought. Negative SDI values denote dry conditions, and the same severity thresholds used for the SPI are applied for categorisation.

\subsection{Normalised Difference Vegetation Index (NDVI)}

The NDVI is a remote-sensing indicator widely used to monitor vegetation health and stress, including agricultural drought. It is derived from the difference in reflectance between the near-infrared (NIR) and red spectral bands:
\[
    \text{NDVI} = \frac{\rho_{\text{NIR}} - \rho_{\text{RED}}}{\rho_{\text{NIR}} + \rho_{\text{RED}}},
\]
where $\rho_{\text{NIR}}$ and $\rho_{\text{RED}}$ are the surface reflectances in their respective bands. Values range from $-1$ to $+1$, with higher values indicating healthy, active vegetation, while lower values reflect stressed or sparse vegetation~\cite{ndvi_seminal_paper}.

\subsection{Discretisation of Indices}
The SPI, SDI, and NDVI are all continuous. However, the proposed model requires discrete inputs. Accordingly, each index was discretised into categorical bins based on thresholds widely used in literature. For example, SPI values are often classified into categories such as “extremely dry,” “moderately dry,” and “normal.” This discretisation not only aids in model implementation but it also motivates interpretability. Table~\ref{tbl:discretisation} below illustrates the bins used:
\begin{table}[h]
    \centering
    \caption{Discretisation thresholds for drought indices.}
    \begin{minipage}{0.48\textwidth}
        \centering
        \begin{tabular}{lcc}
            \toprule
            \textbf{Category} & \textbf{SPI / SDI}  \\
            \midrule
            Severe Drought    & $\leq -1.5$        \\
            Moderate Drought  & $-1.5 < x \leq -0.5$ \\
            Normal            & $-0.5 < x < 0.5$  \\
            Moderate Wet      & $0.5 \leq x < 1.5$ \\
            Severe Wet        & $\geq 1.5$        \\
            \bottomrule
        \end{tabular}
    \end{minipage}%
    \hfill
    \begin{minipage}{0.48\textwidth}
        \centering
        \begin{tabular}{lcc}
            \toprule
            \textbf{Category} & \textbf{NDVI} \\
            \midrule
            Bare soil / water    & $-1 < x <  0.1$   \\
            Sparse vegetation  & $0.1 \leq x < 0.2 $\\
            Moderate vegetation            & $0.2 < x < 0.4$    \\
            Dense vegetation      & $0.4 \leq x < 0.6$ \\
            High density vegetation        & $ 0.6 \leq x < 1 $          \\
            \bottomrule
        \end{tabular}
    \end{minipage}
    \label{tbl:discretisation}
\end{table}


\section{Model Development}
\label{sec:model_development}
\subsection{Model Design}
\label{sec:model_design}

\subsubsection{Defining the Random Variables}

The proposed DNBC is constructed in a general form with $N$ input variables observed across $T$ discrete time steps. All random variables (RVs) in the model are treated as discrete.  

The first set of RVs corresponds to the latent drought states at each time step, denoted by
\[
S_t \in \{1,2,\dots,m\}, \quad t = 1, \dots, T,
\]
where $m$ represents the number of possible drought states. This value of $m$ is not fixed, but will rather be determined via model selection.  

The second set of RVs corresponds to the observed input variables, denoted by
\[
A_t^{(n)} \in \{1,2,\dots,C_n\}, \quad n = 1, \dots, N, \quad t = 1, \dots, T,
\]
where $C_n$ is the cardinality of the $n$-th input variable. In this project, these inputs are the indices used to represent different aspects of drought:
\[
\text{SPI} \equiv A_t^{(1)}, \quad 
\text{SDI} \equiv A_t^{(2)}, \quad 
\text{NDVI} \equiv A_t^{(3)}.
\]
These observed indices constitute the data set $\mathcal{D}$.  

For clarity, we define the following notation which will be used throughout the model formulation:
\[
\vec{S}_{1:T} = \{S_1, S_2, \dots, S_T\}, \quad 
A_{1:T} = \{\vec{A}_1, \vec{A}_2, \dots, \vec{A}_T\},
\]
where each $\vec{A}_t = \{A_t^{(1)}, A_t^{(2)}, \dots, A_t^{(N)}\}$.  

The total number of latent state nodes is therefore $T$, while the number of observed input nodes is $T \times N$. A summary of the random variables, their number of nodes, and their cardinality is provided in Table~\ref{tab:RVs}.

\begin{table}[H]
\centering
\caption{Summary of random variables in the model}
\label{tab:RVs}
\begin{tabular}{lcc}
\hline
\textbf{Name} & \textbf{Number of Nodes} & \textbf{Cardinality} \\ \hline
Latent drought state $S_t$ & $T$ & $m$ \\
General input variable $A_t^{(n)}$ & $T \times N$ & $C_n$ \\
\hline
\end{tabular}
\end{table}

\subsubsection{Graphical Structure \& Assumptions}

Figure~\ref{fig:dnbc-diagram} below displays the model diagram for a DNBC for $T$ time steps and $N$ input variables.  

\begin{figure}[!h]
    \centering
    \includegraphics[width=\linewidth]{dnbc-diagram.png}
    \caption[DNBC Model Diagram]{
The Dynamic Naive Bayes Classifier (DNBC) can be represented as a Bayesian network unfolding over time. At each time step $t$, a latent drought state $S_t$ is modelled as a discrete random variable that governs the latent structure, while the observed input variables $\vec{A}_t = \{A_t^{(1)}, A_t^{(2)}, \dots, A_t^{(N)}\}$ are each solely dependent on $S_t$}
    \label{fig:dnbc-diagram}
\end{figure}

It is important to note the inherent limitations of this model, that being:
\begin{enumerate}[label=(\roman*)]
    \item The dynamic process of the state sequence $S_t$ follows a first-order Markov chain. This means the state at time $t+1$ is conditionally dependent only on the state at time $t$. \label{item:assumption_1}
    \item The dynamic process is stationary, implying that the transition probabilities between states are constant over time. \label{item:assumption_2}
    \item For each time step $t$, the model assumes conditional independence among the input variables $\vec{A}_t$ given the corresponding hidden drought state $S_t$. \label{item:assumption_3}
\end{enumerate}

\subsubsection{Joint Distribution}

The joint probability distribution of the observed variables and latent states in the DNBC can be expressed as:
\begin{equation}
    \begin{align}
        p(S_1, &S_2, \dots, S_T, A^{(1)}_1, A^{(2)}_1, \dots, A^{(N)}_1, A^{(1)}_2, \dots, A^{(N)}_T) \\ 
        &= p(S_1, S_2, \dots, S_T, \vec{A}_1, \vec{A}_2, \dots, \vec{A}_T) \\
        &= p(\vec{S}_{1:T}, A_{1:T}) \\
        &= p(S_1) \cdot \prod\limits_{t=1}^{T-1} p(S_{t+1} \mid S_t) \cdot \prod\limits_{n=1}^{N} \prod\limits_{t=1}^T p(A^{(n)}_t \mid S_t)
    \end{align}
    \label{eqn:joint_distr}
\end{equation}

The following factorisation is possible due to Assumption~\ref{item:assumption_3} of the DNBC and will become useful at a later stage.
\begin{equation} 
    \begin{align} 
        p(\vec{A}_{t} \mid S_t) &= p(A_t^{(1)}, A_t^{(2)}, \dots, A_t^{(N)} \mid S_t) \\ 
        &= p(A_t^{(1)} \mid S_t)p(A_t^{(2)} \mid S_t)\dotsp(A_t^{(N)} \mid S_t) \\ 
        &= \prod\limits_{n=1}^N p(A_t^{(n)} \mid S_t) 
    \end{align} 
    \label{eqn:attribute_rv_factorisation} 
\end{equation}

\subsubsection{Parameterising the Model}
The DNBC is fully specified by three sets of parameters, that being the prior, transition and emission probabilities. 

\paragraph{Prior Probabilities:}  
The initial distribution over the latent drought state $S_1$. \\
The factor table for the priors is show below in Table~\ref{tbl:priors_factor_table}:
\begin{table}[!h]
    \mytable
    \caption{Priors Factor Table}
    \begin{array}{c | c}
        S_1 & p(S_1) \\ 
        \hline
        1 & \pi_1 \\ 
        2 & \pi_2 \\ 
        \vdots & \vdots \\
        m & \pi_m \\ 
    \end{array} 
    \label{tbl:priors_factor_table}
\end{table}

where $\pi_i$ is the probability that the system begins in state $i$. 
\[
\pi_i \equiv p(S_1 = i),
\]

\paragraph{Transition Probabilities:}  
Defines the likelihood of moving to a new hidden state given the current hidden state. \\
The factor table as well as the transition matrix $P^1$ is shows below in Table~\ref{tbl:transition_factor_table}:

\begin{table}[!h]
    \mytable
    \caption{Transition Factor Table \& Transition Matrix}
        \begin{array}{ccc}
        \begin{array}{c c | c}
        S_t & S_{t+1} & p(S_{t+1} \mid S_t) \\ 
        \hline
        1 & 1  & a_{1,1} \\ 
        1 & 2  & a_{1,2} \\ 
        \vdots & \vdots  & \vdots \\
        1 & m  & a_{1, m} \\ 
        2 & 1  & a_{2, 1} \\ 
        2 & 2  & a_{2, 2} \\ 
        \vdots & \vdots  & \vdots \\
        m & m  & a_{m,m} \\ 
        \end{array} 
        &
        \equiv
        &
        P^1 = 
        \begin{bmatrix}
        a_{1,1} & a_{1,2} & \dots & a_{1,m} \\
        a_{2,1} & a_{2,2} & \dots & a_{2,m} \\
        \vdots & \vdots & \ddots & \vdots \\
        a_{m,1} & a_{m,2} & \dots & a_{m,m} \\
        \end{bmatrix}
        \end{array} 
    \label{tbl:transition_factor_table}
\end{table}

Here, $a_{i,j}$ represents the probability of transitioning from state $i$ at time $t$ to state $j$ at time $t+1$. 
\[
a_{i,j} \equiv p(S_{t+1} = j \mid S_t = i).
\]
Note as well that the transition matrix's rows sum to $1$, ie. $\sum_{j=1}^m a_{i,j} = 1$ for all $i$.  

\newpage
\paragraph{Emission Probabilities:}  
Defines the likelihood of observing a particular input variable, given that the system is in a specific hidden state. These parameters encode how the drought indicators behave under each latent drought state.  \\
Once again, the factor table for the emission probabilities is show below in Table~\ref{tbl:emission_factor_table}
\begin{table}[!h]
    \mytable
    \caption{Emission Factor Table}
        \begin{array}{c c | c}
        A^{(n)}_t & S_t & p(A^{(n)}_t \mid S_t) \\ 
        \hline
        1 & 1  & b_1^{(n)}(1) \\ 
        1 & 2  & b_2^{(n)}(1) \\ 
        \vdots & \vdots  & \vdots \\
        1 & m  & b_m^{(n)}(1) \\ 
        2 & 1  & b_1^{(n)}(2) \\ 
        2 & 2  & b_2^{(n)}(2) \\ 
        \vdots & \vdots  & \vdots \\
        C_n & m  & b_m^{(n)}(C_n) \\ 
        \end{array} 
    \label{tbl:emission_factor_table}
\end{table}

Where, $b_i^{(n)}(j)$ is the likelihood of observing input variable $n$ take on the value $j$, given that its corresponding hidden drought state is equal to $i$
\[
b_i^{(n)}(j) \equiv p(A_t^{(n)} = j \mid S_t = i).
\]


Taken together, the parameter set fully determines the DNBC. It is important to note that due to the parameters being time independent, as the model assumes stationarity, the rules governing drought state transitions and emissions are invariant across time.

\subsection{Inference}
\label{sec:inference}

In this section, inference for the DNBC is developed under the assumption that the parameters $\Theta$ are known and the input variables $A_{1:T}$ are observed. Since the attributes are not random at this stage, the task becomes trying to infer the distribution of the hidden drought states:
\[
    p(\vec{S}_{1:T} \mid A_{1:T}, \Theta),
\]
This will later be used for the E-step in the EM algorithm.

The inference procedure is carried out using the Junction Tree (JT) framework, which provides exact inference. Messages are propagated through the tree, beginning at the leaf clusters and moving inward~\cite{lauritzen1988local}. \\
Figure~\ref{fig:jt_diagram} illustrates the JT structure associated with the DNBC.

\begin{figure}[H]
    \centering
    \includegraphics[width=\linewidth]{jt-diagram.png}
    \caption[Junction Tree Diagram]{Junction Tree representation of the DNBC. Each cluster groups together latent state variables and observed attributes, with sepsets defined along the edges. Messages are propagated through the tree to perform exact inference.}
    \label{fig:jt_diagram}
\end{figure}

It is useful to note the factorisation of the observed attribute RVs at Equation~\ref{eqn:attribute_rv_factorisation}. The cluster potentials are summarised in Table~\ref{tbl:cluster_potentials}.
\begin{table}[!h]
    \mytable
    \caption[Cluster Potentials Of Junction Tree]{Cluster potentials for the DNBC. Each potential corresponds either to a state transition or to a state-attribute relationship.}
    \begin{aligned}[c]
        &- \\
        \psi_2(S_1, S_2) &= p(S_2 \mid S_1) \\
        &\vdots \\
        \psi_t(S_{t-1}, S_t) &= p(S_t \mid S_{t-1}) \\
        &\vdots \\
        \psi_T(S_{T-1}, S_T) &= p(S_T \mid S_{T-1}) \\
    \end{aligned}
    \qquad \qquad \qquad
    \begin{aligned}[c]
        \psi_1(S_1, \vec{A}_1) &= p(S_1)p(\vec{A}_1 \mid S_1) \\
        \psi_2(S_2, \vec{A}_2) &= p(\vec{A}_2 \mid S_2) \\
        &\vdots \\
        \psi_t(S_t, \vec{A}_t) &= p(\vec{A}_t \mid S_t) \\
        &\vdots \\
        \psi_T(S_T, \vec{A}_T) &= p(\vec{A}_T \mid S_T) \\
    \end{aligned}
    \label{tbl:cluster_potentials}
\end{table}

\subsubsection{Message Passing}
Messages are defined between clusters, with sepsets given by the product of messages between clusters. 

\paragraph{Upward messages:}  
Because the attributes are observed, upward messages collapse to the corresponding likelihood terms.
\begin{align*}
    \delta_{t\uparrow} (S_t) &= \sum\limits_{\vec{A}_t} \psi_t(S_t, \vec{A}_t) \\ 
    &= \sum\limits_{\vec{A}_t} p(\vec{A}_t \mid S_t) \\
    &= p(\vec{A}_t \mid S_t)
\end{align*}
since marginalisation over the observed attributes reduces to their likelihood.


\paragraph{Rightward messages:}  
Rightward propagation starts at the leftmost cluster and moves forward in time:
\begin{align*} 
    \delta_{1 \rightarrow 2} (S_1) &= \sum\limits_{\vec{A}_1} \psi_1(S_1, \vec{A}_1) \\
    &= \sum\limits_{\vec{A}_1} p(S_1)p(\vec{A}_1 \mid S_1) \\
    &= p(S_1)p(\vec{A}_1 \mid S_1) \numberthis \label{eqn:rightward_message_init} \\
    \\
    \delta_{t \rightarrow t+1} (S_t) &= \sum\limits_{S_{t-1}} \psi_t(S_{t-1}, S_t) \delta_{t-1 \rightarrow t}(S_{t-1}) \delta_{t\uparrow}(S_t) \\
    &= \sum\limits_{S_{t-1}} p(S_t \mid S_{t-1}) \delta_{{t-1} \rightarrow t}(S_{t-1}) p(\vec{A}_t \mid S_t) \\
    &= p(\vec{A}_t \mid S_t) \sum\limits_{S_{t-1}} p(S_t \mid S_{t-1}) \delta_{{t-1} \rightarrow t}(S_{t-1}) \numberthis \label{eqn:rightward_message_final} \\
\end{align*}

\paragraph{Leftward messages.}  
Similarly, leftward propagation begins at the final cluster and proceeds backward:
\begin{align*} 
    \delta_{T-1 \leftarrow T} (S_{T-1}) &= \sum\limits_{S_T} p(S_T \mid S_{T-1})p(\vec{A}_T \mid S_T), \numberthis \label{eqn:leftward_message_init} \\
    \\
    \delta_{t-1 \leftarrow t} (S_{t-1}) &= \sum\limits_{S_t} p(S_t \mid S_{t-1}) \delta_{t \leftarrow t+1}(S_t) p(\vec{A}_t \mid S_t). \numberthis \label{eqn:leftward_message_final}
\end{align*}

\subsubsection{Remarks}
In this framework, the clusters of primary interest are $\psi_t(S_t, S_{t+1})$ and the sepsets $\mu_{t,t+1}(S_t)$, which directly contribute to the computation of $p(\vec{S}_{1:T} \mid A_{1:T}, \Theta)$. As a result, downward messages (e.g., from $\psi_t(S_{t-1}, S_t)$ to $\psi_t(S_t, \vec{A}_t)$) are not of interest. \\
Finally, it is worth noting that for JTs, since the underlying graph is a tree, message passing is exact. We follow a specific message-passing ordering of the standard Belief Propagation algorithm, which is guaranteed to converge to the exact marginals.

\subsubsection{Forward–Backward Algorithm}
At this point, it is natural to highlight the connection between the JT approach described above and the more classical algorithms for HMMs along with their variants. Readers familiar with the literature will recognise that the message passing operations we performed are precisely the equivalent to the well-known \emph{forward–backward equations}~\cite{binder1997space,wiki:forward_backward,aviles}. \\
The forward and backward recursions applied to the proposed model are shown below:

\paragraph{Forward:}
\begin{flalign}
    &\qquad \text{Define:} \qquad \alpha_t^k = p(A_{1:t}, S_t = i)& \nonumber \\
    &\qquad \text{Init:} \qquad \alpha_1^k = p(S_1 = k)p(\vec{A}_1 \mid S_1 = k)& \nonumber \\
    &\qquad \text{Iteration:} \qquad \alpha_t^k = p(\vec{A}_t \mid S_t = k)\sum\limits_{i=1}^m \alpha_{t-1}^i \cdot p(S_t = k \mid S_{t-1} = i)& \label{eqn:forward_eqn}
\end{flalign}

\paragraph{Backward:}
\begin{flalign}
    &\qquad \text{Define:} \qquad \beta^k_t = p(A_{1:t}, S_t = i)& \nonumber \\
    &\qquad \text{Init:} \qquad \beta^k_T = 1 \quad \forall \, k& \nonumber \\
    &\qquad \text{Iteration:} \qquad \beta^k_t = \sum\limits_{i=1}^m p(S_{t+1} = i \mid S_t = k) \cdot p(\vec{A}_{t+1} \mid S_{t+1} = i) \cdot \beta_{t+1}^i& \label{eqn:backward_eqn}
\end{flalign}

\subsubsection*{Remarks on Forward–Backward and Baum–Welch}
The messages passed in the JT (Equations~\ref{eqn:rightward_message_init} - \ref{eqn:leftward_message_final}) coincide with the $\alpha$ and $\beta$ recursions in Equations~\ref{eqn:forward_eqn}–\ref{eqn:backward_eqn}. The distinction is thus in presentation alone. The JT framework is a generalisation for arbitrary graphical models, whereas the forward–backward is the special case formulation for the structure of HMMs\cite{hmm_slides}.

It is worth emphasising the parallel between the JT messages and the forward–backward quantities. The forward recursion $\alpha_t^k=p(A_{1:t},S_t=k)$ and the backward recursion $\mathit{\beta_t^k=p(A_{t+1:T}\mid S_t=k)}$ are algebraically equivalent to the inward and outward sum–product messages in the JT~\ref{fig:jt_diagram}. When inward and outward messages are combined at a cluster or sepset, the resulting posterior marginals $p(S_t \mid A_{1:T})$ and pairwise marginals $p(S_t,S_{t+1}\mid A_{1:T})$ coincide with the responsibilities computed from Baum-Welch. Thus, the JT message-passing procedure and the forward–backward algorithm produce identical posterior marginals. These results will be of interest in the following sub section for parameter estimation~\cite{aviles,dnbc_drought_first,hmm_slides,wiki:baum_welch}.

In summary, the JT formulation highlights the structural perspective, while forward–backward and Baum–Welch remain the traditional algorithms in the literature. Both views are mathematically equivalent and lead to the same computations.

\subsection{Parameter Estimation}
\label{sec:param_estimation}
Parameter estimation for the DNBC is carried out using the Expectation–Maximization (EM) algorithm~\cite{moon_tk}.  
We distinguish between the hidden variables, observed data, and model parameters as follows:

\begin{align*}
    \mathcal{H} &= (S_t)_{t=1}^T  \\
    \mathcal{D} &= (\vec{A}_t)_{t=1}^T  \\
    \Theta &= (\vec{\theta}_1, \vec{\theta}_2, \vec{\theta}_3)  \\
    \text{Whe}& \text{re,} \\
     & \quad \vec{\theta}_1 = \{\pi_1, \pi_2, \dots, \pi_m\} \equiv \text{Priors Probabilities}  \\
     & \quad \vec{\theta}_2 = \{a_{i,j} \mid i,j = 1, \dots, m\} \equiv \text{Transition Probabilities}  \\
     & \quad \vec{\theta}_3 = \left\{ b_i^{(n)}(j) \,\middle|\, i = 1,\dots,m; \, n = 1,\dots,N; \, j = 1,\dots,C_n \right\} \equiv \text{Emission Probabilities} 
\end{align*}

The EM algorithm iteratively alternates between two steps:

\subsubsection{1. E-Step}

In this step, we hold $\Theta$ fixed and compute the posterior distribution over the hidden states:

\begin{align}
    q(\mathcal{H}) &= p(\mathcal{H} \mid \mathcal{D}, \Theta) \nonumber \\
    &= p(\vec{S}_{1:T} \mid A_{1:T}, \Theta) \numberthis \label{eqn:e_step}
\end{align}

This corresponds directly to the inference problem, as previously discussed in Section~\ref{sec:inference}.

\subsubsection{2. M-Step}

Next, with $q$ fixed, we maximise the variational lower bound
\[
    \mathscr{L}(q, \Theta) = \sum\limits_{\mathcal{H}}q(\mathcal{H}) \cdot \log \left( \frac{p(\mathcal{D}, \; \mathcal{H} \mid \Theta)}{q(\mathcal{H})} \right)
\]
with respect to $\Theta$.

Equivalently, this requires solving
\begin{align}
    \Theta &= \underset{\Theta}{\operatorname{argmax}} \; \mathcal{Q}(\Theta) \nonumber \\
    &= \underset{\Theta}{\operatorname{argmax}} \sum\limits_{\mathcal{H}} q(\mathcal{H}) \cdot \log \, p(\mathcal{D}, \mathcal{H} \mid \Theta) \numberthis \label{eqn:m_step}
\end{align}

The inner term, $\log \, p(\mathcal{D}, \mathcal{H} \mid \Theta)$, is simply the log of the joint distribution introduced in Equation~\ref{eqn:joint_distr}. Expanding this expression yields:
\begin{align*}
    p(A_{1:T} \, \vec{S}_{1:T} \mid \Theta) &= \log \, p(S_1 \mid \vec{\theta}_1) \\ 
    &\qquad + \sum\limits_{t=1}^{T-1} \log \, p(S_{t+1} \mid S_t, \, \vec{\theta}_2) \\ 
    &\qquad + \sum\limits_{n=1}^N \sum\limits_{t=1}^T \log \, p(A_t^{(n)} \mid S_t, \, \vec{\theta}_3) 
\end{align*}

Substituting this into $\mathcal{Q}(\Theta)$ and carefully reorganising terms allows us to isolate contributions from priors, transitions, and emissions. Since all RVs are discrete, probabilities translate directly into parameterised forms, and the optimisation decouples naturally across $\vec{\theta}_1, \vec{\theta}_2$, and $\vec{\theta}_3$.

\begin{align*}
    \mathcal{Q} &= \sum\limits_{\mathcal{H}} q(\mathcal{H}) \cdot \log \, p(\mathcal{D}, \mathcal{H} \mid \Theta) \\
    &= \sum\limits_{\mathcal{H}} q(\mathcal{H}) \cdot \bigr[ \log \, p(S_1 \mid \vec{\theta}_1) \\ 
    &\qquad \qquad \qquad + \sum\limits_{t=1}^{T-1} \log \, p(S_{t+1} \mid S_t, \, \vec{\theta}_2) \\ 
    &\qquad \qquad \qquad + \sum\limits_{n=1}^N \sum\limits_{t=1}^T \log \, p(A_t^{(n)} \mid S_t, \, \vec{\theta}_3) \bigr]
\end{align*}

We then multiply $\sum\limits_{\mathcal{H}} q(\mathcal{H})$ through, understanding that $\mathcal{H} = (S_1, \dots, S_T)$
\begin{align*}
    &= \sum\limits_{S_1, \dots, S_T} \log \, p(S_1 \mid \vec{\theta}_1) \, q(S_1, \dots, S_T) \\ 
    & \qquad + \sum\limits_{S_1, \dots, S_T} \sum\limits_{t=1}^{T-1} \log \, p(S_{t+1} \mid S_t, \, \vec{\theta}_2) \, q(S_1, \dots, S_T) \\ 
    &\qquad + \sum\limits_{S_1, \dots, S_T} \sum\limits_{n=1}^N \sum\limits_{t=1}^T \log \, p(A_t^{(n)} \mid S_t, \, \vec{\theta}_3) \, q(S_1, \dots, S_T) \\
    \\
    &= \sum\limits_{S_1} \log \, p(S_1 \mid \vec{\theta}_1) \, q(S_1) + \sum\limits_{S_2, \dots, S_T} q(S_2, \dots, S_T) \\ 
    & \qquad + \sum\limits_{t=1}^{T-1} \sum\limits_{S_t, S_{t+1}} \log \, p(S_{t+1} \mid S_t, \, \vec{\theta}_2) \, q(S_t, S_{t+1}) + \sum_{\substack{S_1,\dots, S_T \\ \setminus S_t, \, S_{t+1}}} q(S_1, \dots, S_T) \\ 
    &\qquad + \sum\limits_{n=1}^N \sum\limits_{t=1}^T \sum\limits_{S_t} \log \, p(A_t^{(n)} \mid S_t, \, \vec{\theta}_3) \, q(S_t) + \sum_{\substack{S_1,\dots, S_T \\ \setminus S_t}} q(S_1, \dots, S_T) \\
\end{align*}

Since the goal is to optimise w.r.t $\Theta = (\vec{\theta}_1,\vec{\theta}_2,\vec{\theta}_3)$, all the terms not involving $\Theta$ can be dropped, whilst also using the result $\sum\limits_{S_t} p(S_t) = \sum\limits_{i=1}^m p(S_t = i)$
\begin{align*}
    &= \sum\limits_{i=1}^m \log \, p(S_1 = i \mid \vec{\theta}_1) \, q(S_1 = i) \\
    & \qquad + \sum\limits_{t=1}^{T-1} \sum\limits_{i=1}^m \sum\limits_{j=1}^m \log \, p(S_{t+1} = j \mid S_t = i, \, \vec{\theta}_2) \, q(S_t = i, S_{t+1} = j) \\
    &\qquad + \sum\limits_{n=1}^N \sum\limits_{t=1}^T \sum\limits_{i=1}^m \log \, p(A_t^{(n)} \mid S_t = i, \, \vec{\theta}_3) \, q(S_t = i) \\
\end{align*}

This representation can be expressed in terms of the model parameters. Because all random variables are discrete, the probabilities naturally reduce to combinations of these parameters.
\begin{align*}
    &= \sum_{i=1}^m q(S_1 = i) \log \pi_i \\
    &\qquad + \sum_{t=1}^{T-1}\sum_{i=1}^m\sum_{j=1}^m q(S_t=i,S_{t+1}=j)\log a_{i,j}\\ 
    &\qquad +  \sum\limits_{t=1}^T \sum\limits_{i=1}^m  q(S_t = i)  \sum\limits_{n=1}^N \log b_i^{(n)}(A_t^{(n)}) \\
\end{align*}

Each of the target parameters are now separated into their own terms and thus can be easily optimised in isolation. This yields the standard re-estimation updates~\cite{jm3,xing_slides}:
\begin{equation}
    \boxed{\;\pi_i^{\text{new}} = q(S_1=i)\;}
    \label{eqn:prior_update_rule}
\end{equation}
\begin{equation}
    \boxed{\;a_{i,j}^{\text{new}}=\frac{\sum\limits_{t=1}^{T-1} q(S_t=i,S_{t+1}=j)}{\sum\limits_{t=1}^{T-1} q(S_t=i)}\;}
    \label{eqn:transition_update_rule}
\end{equation}
\begin{equation}
    \boxed{\; b_i^{(n)}(j)^{\text{new}} = \frac{\sum\limits_{t=1}^T q(S_t = i) \cdot \mathbf{1}(A_t^{(n)} = j)}{\sum\limits_{t=1}^T q(S_t = i)} \;}
    \label{eqn:emission_update_rule}
\end{equation}

\subsection{Model Selection}
\label{sec:model_selection}
Model selection will involve determining the appropriate cardinality of each latent drought states $S_t$, that is, determining the value of $m$. When selecting $m$, a balance must be struck between model complexity and goodness of fit, as a larger value of $m$ gives the model a greater ability to capture subtle drought dynamics but risks overfitting. On the other hand, a smaller number may be too restrictive to reflect the underlying processes.

To guide this choice, three complementary criteria are applied: the Akaike Information Criterion (AIC), the Bayesian Information Criterion (BIC), and the maximised log-likelihood of the fitted model. These are given by
\begin{align}
    AIC &= -2 \cdot \log L(\Theta) + 2p, \label{eqn:aic}\\
    BIC &= -2 \cdot \log L(\Theta) + p \cdot \log k, \label{eqn:bic}
\end{align}
where $L(\Theta)$ is the maximised value of the likelihood function, $p$ is the number of free parameters in the model, and $k$ is the number of data points.  

The philosophy underlying these criteria is rooted in Occam’s razor, which is often phrased as ”the simplest
explanation is usually the best one”. AIC and BIC both balance model fit against complexity, but with differing severity. BIC applies a stronger penalty on complexity and is thus generally considered more consistent with Occam’s razor~\cite{Barber_2012}. Thus, the framework for selecting $m$ as follows:
\begin{enumerate}
    \item \textbf{Primary:} select the model with the lowest BIC, penalising unnecessary complexity.
    \item \textbf{Secondary:} use AIC to cross-check results.
    \item \textbf{Tertiary:} inspect the log-likelihood curve. If $\log L(\Theta)$ improves only marginally as $m$ increases, the simpler model is preferred (the so-called “elbow rule”).
\end{enumerate}

In practice, model selection is performed by sweeping across candidate values of $m$, fitting a model for each case, and comparing their AIC, BIC, and log-likelihood values. The final choice of $m$ seeks to minimise both AIC and BIC while ensuring that the likelihood $L(\Theta)$ does not deteriorate substantially.

\subsubsection{Choice of $k$}

The term $k$ in~\eqref{eqn:bic} represents the number of data points. Following common practice in the literature and implementation libraries such as the \texttt{seqHMM} package in R~\cite{cran_hmmSeq}, $k$ is calculated as: 
\[
    k = T \times N,
\]

\subsubsection{Number of Free Parameters $p$}

The number of free parameters $p$ corresponds to the model’s degrees of freedom. This includes contributions from the prior probabilities~\ref{tbl:priors_factor_table}, the transition probabilities~\ref{tbl:transition_factor_table}, and the emission probabilities~\ref{tbl:emission_factor_table}. It is widely accepted in the literature and implementations regarding HMMs and its variants~\cite{cran_hmmSeq,free_params_slides} that
\begin{align*}
    p &= (m-1) + m(m-1) + \sum_{n=1}^N m(C_n - 1) \\
      &= m^2 - 1 + m \sum_{n=1}^N (C_n - 1),
\end{align*}

\subsubsection{Log-Likelihood Estimation}

The third component of model selection is the log-likelihood, $\ell(\Theta)$, which measures the probability of the observed data under the model parameters:
\[
    \ell(\Theta) = p(A_{1:T}^\text{obs} \mid \Theta).
\]

This likelihood can be evaluated efficiently using the forward algorithm. Recall that the forward variable is defined as
\[
    \alpha_t^k = p(S_t = k, A_{1:t} \mid \Theta),
\]
The overall likelihood is then simply obtained by marginalising over the latent state at the final time step:
\begin{align*}
    \sum_{i=1}^m \alpha_T^i &= \sum_{S_T} p(A_{1:T}, S_T \mid \Theta) \\
                             &= p(A_{1:T} \mid \Theta) \;=\; \ell(\Theta).
\end{align*}

Although the derivation via the forward algorithm is given for clarity, it has been established that it is equivalent to the rightward message-passing procedure in the JT approach (Section~\ref{sec:inference}). For this approach, an additional downward message $\delta_{\downarrow T}(S_T)$ must be computed at the final cluster to obtain the posterior $\psi_T(S_T, \vec{A}_T)$. Marginalising out $S_T$ from this posterior yields the desired likelihood. It is useful to see Figure~\ref{fig:jt_diagram}.

Create a final draft for this:

\section{Model Implementation}
\label{sec:model_implementation}

\subsection{Programming Environment \& Tools}
All aspects of the DNBC model were implemented in \texttt{C++}, primarily chosen for its computational efficiency and the availability of the \texttt{emdw} library. This library provides robust functionality for probabilistic graphical models. The \texttt{C++} implementation handled the construction of factors, junction tree message passing, parameter estimation, model selection, and extraction of posterior outputs.  

Python was used to complement this workflow, particularly for data-related tasks such as raw data extraction, preprocessing into model inputs, postprocessing of model outputs, and visualisation of results. This division allowed \texttt{C++} to focus on core model computation while Python streamlined data management and analysis.

\subsection{Data Pipeline Implementation}
The data pipeline was designed to translate raw climate data into discretised indices that serve as inputs to the model. The visualisation of this pipeline and its flow is shown in Figure~\ref{fig:pipeline-diagram}
At a high level, the process consisted of:

\begin{figure}[!h]
    \centering
    \includegraphics[width=\linewidth]{pipeline-diagram.png}
    \caption[Data Pipeline For DNBC Inputs]{Data pipeline from raw climate and vegetation data to discretised indices used as DNBC inputs. The pipeline consists of five stages: data collection, preprocessing, index calculation, discretisation, and input formatting.}
    \label{fig:pipeline-diagram}
\end{figure}

\begin{enumerate}
    \item \textbf{Data Collection:} Acquiring raw climate and vegetation data.
    \item \textbf{Preprocessing:} Handle missing values, time and space alignment, ensuring data consistency, etc.
    \item \textbf{Index Calculation:} Input indices (SPI, SDI, and NDVI) were calculated following formulas given in Section~\ref{sec:index-calc}.
    \item \textbf{Discretisation:} Convert continuous indices into categorical for DNBC input.
    \item \textbf{Input Formatting:} Finally, these discretised indices are formatted and saved as a CSV file, ready for model ingestion with \texttt{C++}.
\end{enumerate}
This pipeline was implemented using Python.

\subsection{Model Implementation}
Implementation of the DNBC was achieved through two main functions: \texttt{runEM} and \texttt{modelSelection}.

\subsubsection*{runEM}
The \texttt{runEM} function performed parameter estimation using the EM algorithm, paired with exact inference offered by the JT methodology:
\begin{enumerate}
    \item Random initialisation of parameters (sampled from a Gaussian distribution, typically standard normal).
    \item Construction of discrete factors using the \texttt{emdw} library.
    \item Initialisation of cluster potentials and message passing as seen in Figure~\ref{fig:jt_diagram} to perform exact inference.
          \begin{itemize}
              \item To avoid underflow, all factors were normalised after each update. This sacrificed some efficiency but significantly improved numerical stability.
          \end{itemize}
    \item Parameter update step (M-step).
    \item Likelihood calculation and convergence check using the relative tolerance criterion:
    \[
        \frac{|\ell(\Theta)^{\text{new}} - \ell(\Theta)^{\text{old}}|}{\ell(\Theta)^{\text{old}}} < \varepsilon
    \]
        where the maximum number of iterations was capped at 100, whilst the threshold value was chosen to be $\epsilon = 10^{-4}$.
\end{enumerate}

\subsubsection*{modelSelection}
The \texttt{modelSelection} function evaluated different values of the hyperparameter $m$:
\begin{enumerate}
    \item For each candidate $m$, the model was run with 10 random restarts.
    \item The best run (highest log-likelihood) was retained.
    \item Model fit metrics (AIC, BIC, and maximum log-likelihood) were recorded for each $m$.
    \item Results were exported to CSV files for analysis with Python.
\end{enumerate}

\subsection{Model Output}

The final model outputs were exported in two forms:
\begin{itemize}
    \item Posterior decoding using the Maximum Posterior Marginal (MPM) rule.
    \item State sequence decoding using the Viterbi algorithm.
\end{itemize}

\subsubsection{1. Maximum Posterior Marginal (MPM) rule}

The MPM rule involves computing the point-wise marginal for latent drought state $S_t$:
\[
    \hat{s}_t = \underset{s}{\operatorname{argmax}} \;\; p(S_t = s \mid A_{1:T}, \Theta)
\]
This is easily obtained using the results from Section~\ref{sec:inference}.

Conceptually, this rule will pick the most likely state, with a confidence attached, at each time step independently. This is particularly useful if the goal is signal labeling. It should be noted that this rule often leads to an unlikely or even impossible state sequence (eg., $S_t \equiv$ Very Wet \|\; $S_{t+1} \equiv$ Very Dry) and thus is only sensible to use when temporal consistency can be ignored. 

\subsubsection{2. Viterbi Algorithm}
On the other hand, the Viterbi algorithm produces a temporally coherent sequence that respects the state transition dynamics. Mathematically, the Viterbi decoding produces the single most probable joint state sequence:
\[
    \vec{s}^* = \underset{\vec{S}_{1:T}}{\operatorname{argmax}} \;\; p(\vec{S}_{1:T}\mid A_{1:T}, \Theta),
\]
This decoding provides a more realistic representation of drought progression~\cite{viterbi}.  

\subsubsection{Output Format}
Both outputs were exported to CSV files by the \texttt{C++} implementation. Post-processing, visualisation, and evaluation of these outputs were performed in Python, enabling comparison with the original input indices and facilitating analysis.
